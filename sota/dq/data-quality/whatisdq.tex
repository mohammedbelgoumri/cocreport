\subsection{What is data quality?}\label{sub:what_is_data_quality}

A great body of research has been dedicated to making sense of the concept of data quality,
producing an equally great number of views on how to define it.
Among the few common threads between a majority of these works are the following two observations:
\begin{enumerate*}[label=(\arabic*)]
  \item data quality is context-dependent, i.e.,
        rather than being a function of the dataset alone,
        it is a function of the entire context of its use, and
  \item data quality is multidimensional, i.e.,
        multiple aspects of data must be considered when assessing it.
\end{enumerate*}
The first observation is demonstrated by the convergent pragmatic definition of data quality
as ``fitness for use''~\cite{sidiDataQualitySurvey2012},
``suitability for business purposes''~\cite{hagendorffLinkingHumanMachine2021},
or ``the degree to which the data of interest fulfills given requirements''%
~\cite{hassensteinDataQualityConcepts2022}.
The second is supported by the overwhelming trend of using multiple properties of a dataset
to assess its quality~\cite{%
  budachEffectsDataQuality2022,%
  mahantiDataQualityDimensions2019,%
  sidiDataQualitySurvey2012%
}.

For the purpose of this work, we adhere to both of the above-stated observations.
The first one boils down to the fact that
assigning quality to a dataset ---or anything else for that matter---
is a normative judgment, and therefore cannot be done without an underlying objective.
In the context of \gls{abb:ml}, and particularly for risk minimization,
the objective is to minimize the risk \(\risk\).
Although this is usually approximated by \gls{abb:erm},
minimizing \(\risk\) remains the ultimate goal, and thus the measure of quality.
This implies the following principle, which we will use
to guide us to a definition of data quality.

\begin{principle}\label{princple:performance_is_dq}\
  A good dataset \(\xi\) can be defined as one that (when used for \gls{abb:erm})
  produces a good model, which in turn can be defined
  as a model \(\hat{\theta}_{\xi}\) with low \emph{true risk}
  \(\risk\left( \hat{\theta}_{\xi} \right)\).
\end{principle}

Echoes of this principle can be found in the literature,
for example in~\cite{danilovDataQualityEstimation2023,camachoQualityQualityOut2023}.
Its main advantage is that it reduces the nebulous and poorly defined concept of data quality
to the comparatively well-defined concept of model quality,
measured by the completely unambiguous true risk.
The second observation can be captured by the use of multiple properties,
which together imply that \(\risk\left( \hat{\theta}_{\xi} \right)\)
is small.
In the following, we will explore some of these properties.
