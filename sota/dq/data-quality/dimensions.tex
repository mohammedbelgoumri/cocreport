\subsection{Data quality dimensions}%
\label{sub:data_quality_dimensions}

It follows from \Cref{princple:performance_is_dq} that
the \emph{best} dataset \(\xi\) produces the \emph{best}
(or Bayes-optimal) model, i.e., \(\hat{\theta}_{\xi} = \theta^{\ast}\).
More generally, a \emph{good} dataset is one that produces a \emph{good} model,
i.e., \(\hat{\theta}_{\xi} \approx \theta^{\ast}\).
Early works on the theory of \gls{abb:ml}~\cite{%
	vapnikPrinciplesRiskMinimization1991,%
	neweyChapter36Large1994,%
	vapnikNatureStatisticalLearning2010%
}
have provided theoretical guarantees for this to be the case
under certain fairly general conditions
on the hypothesis class \(\mathcal{H}\) and the parameter space \(\Theta\).
More precisely, they have shown that, with high probability,
a large enough dataset \(\xi\), drawn \gls{abb:iid}
from the true distribution \(\Pr_{(X, Y)}\),
will produce a good model (in the sense defined above).

This already gives us three sufficient conditions for a dataset to be good,
namely that it is:
\begin{enumerate*}[label=(\roman*)]
	\item large enough,
	\item \gls{abb:iid}, and
	\item drawn from the true distribution.
\end{enumerate*}
Size is usually not seen as a data quality issue,
so we will not consider it further.
The other two conditions, however, are of great interest to us,
and can serve as the main axes of our definition of data quality.
Based on the previous analysis,
we can tentatively divide \gls{abb:dq} concerns into the following categories:
\begin{enumerate}
	\item \emph{Data dependence}.
	      Which occurs when the elements of \(\xi\) are not independent.
	\item \emph{Distribution mismatch}.
	      Which occurs when the elements of \(\xi\)
	      do not follow the true distribution \(\Pr_{(X, Y)}\).
	      The joint distribution \(\Pr_{(X, Y)}\)
	      is characterized by the marginal distribution \(\Pr_{X}\)
	      and the conditional distribution \(\Pr_{Y \mid X}\),
	      which allows us to distinguish between two types of distribution mismatch:
	      \begin{enumerate}
		      \item \emph{Attribute skew}.
		            Which happens when the sampling process
		            is biased towards certain regions of the input space \(\mathcal{X}\).
		      \item \emph{Label noise}.
		            Which happens when the labeling process is error-prone.
	      \end{enumerate}
\end{enumerate}
It is worth noting that the above scenarios are not mutually exclusive.
Distribution mismatch can happen due to a combination
of attribute skew and label noise,
and data dependence is likely to cause attribute skew.

An easy objection one could raise against the above categorization,
and indeed against \Cref{princple:performance_is_dq} itself,
is its reliance on the premise that model quality
can be reduced to performance (i.e., true risk).
In doing so, it fails to completely capture
the context-dependent nature of \gls{abb:dq}.
As a result, all the above considerations
are more or less \emph{intrinsic} to the data.

To justify the inclusion of \emph{extrinsic} considerations,
we argue that model performance is not sufficient to capture model quality.
A model can be considered bad while having excellent performance,
if it fails to satisfy certain ethical or social criteria.
For example, a model that has near-optimal true risk
can have arbitrarily high risk on a certain subset \(A \subset \mathcal{X}\),
if \(\Pr{\left[X \in A \right]}\) is small enough.
Such a model is unfair to the elements of \(A\),
which can lead to legal and ethical issues if \(A\) is a marginalized group.
Similarly, a model that allows for the reconstruction of \(\xi\),
thereby betraying the privacy of the individuals in the training set
is also considered bad, even if it has excellent performance.
Alternatively, the data could be of low quality
due to its source rather than its content.
For example, a dataset that can include deliberately wrong information.
Accounting for these considerations, we amend \Cref{princple:performance_is_dq}
to a more general form.
\begin{principle}\label{princple:mq_is_dq}
	A good dataset \(\xi\) can be defined as one that (when used for \gls{abb:erm})
	produces a good model, which in turn can be defined
	as a model \(\hat{\theta}_{\xi}\) that satisfies certain ethical and social criteria,
	in addition to having low true risk \(\risk\left( \hat{\theta}_{\xi} \right)\).
\end{principle}
An extended version of the above taxonomy to include these extrinsic considerations
in compliance with \Cref{princple:mq_is_dq} is shown in \Cref{fig:dq:dims:taxonomy}.
Each dimension of this taxonomy will be discussed in detail in  \Crefrange{sec:independence}{sec:trust}.


\begin{figure}[hbt]
	\begin{center}
		% \resizebox{\linewidth}{!}{\subsection{Data quality dimensions}%
\label{sub:data_quality_dimensions}

It follows from \Cref{princple:performance_is_dq} that
the \emph{best} dataset \(\xi\) produces the \emph{best}
(or Bayes-optimal) model, i.e., \(\hat{\theta}_{\xi} = \theta^{\ast}\).
More generally, a \emph{good} dataset is one that produces a \emph{good} model,
i.e., \(\hat{\theta}_{\xi} \approx \theta^{\ast}\).
Early works on the theory of \gls{abb:ml}~\cite{%
	vapnikPrinciplesRiskMinimization1991,%
	neweyChapter36Large1994,%
	vapnikNatureStatisticalLearning2010%
}
have provided theoretical guarantees for this to be the case
under certain fairly general conditions
on the hypothesis class \(\mathcal{H}\) and the parameter space \(\Theta\).
More precisely, they have shown that, with high probability,
a large enough dataset \(\xi\), drawn \gls{abb:iid}
from the true distribution \(\Pr_{(X, Y)}\),
will produce a good model (in the sense defined above).

This already gives us three sufficient conditions for a dataset to be good,
namely that it is:
\begin{enumerate*}[label=(\roman*)]
	\item large enough,
	\item \gls{abb:iid}, and
	\item drawn from the true distribution.
\end{enumerate*}
Size is usually not seen as a data quality issue,
so we will not consider it further.
The other two conditions, however, are of great interest to us,
and can serve as the main axes of our definition of data quality.
Based on the previous analysis,
we can tentatively divide \gls{abb:dq} concerns into the following categories:
\begin{enumerate}
	\item \emph{Data dependence}.
	      Which occurs when the elements of \(\xi\) are not independent.
	\item \emph{Distribution mismatch}.
	      Which occurs when the elements of \(\xi\)
	      do not follow the true distribution \(\Pr_{(X, Y)}\).
	      The joint distribution \(\Pr_{(X, Y)}\)
	      is characterized by the marginal distribution \(\Pr_{X}\)
	      and the conditional distribution \(\Pr_{Y \mid X}\),
	      which allows us to distinguish between two types of distribution mismatch:
	      \begin{enumerate}
		      \item \emph{Attribute skew}.
		            Which happens when the sampling process
		            is biased towards certain regions of the input space \(\mathcal{X}\).
		      \item \emph{Label noise}.
		            Which happens when the labeling process is error-prone.
	      \end{enumerate}
\end{enumerate}
It is worth noting that the above scenarios are not mutually exclusive.
Distribution mismatch can happen due to a combination
of attribute skew and label noise,
and data dependence is likely to cause attribute skew.

An easy objection one could raise against the above categorization,
and indeed against \Cref{princple:performance_is_dq} itself,
is its reliance on the premise that model quality
can be reduced to performance (i.e., true risk).
In doing so, it fails to completely capture
the context-dependent nature of \gls{abb:dq}.
As a result, all the above considerations
are more or less \emph{intrinsic} to the data.

To justify the inclusion of \emph{extrinsic} considerations,
we argue that model performance is not sufficient to capture model quality.
A model can be considered bad while having excellent performance,
if it fails to satisfy certain ethical or social criteria.
For example, a model that has near-optimal true risk
can have arbitrarily high risk on a certain subset \(A \subset \mathcal{X}\),
if \(\Pr{\left[X \in A \right]}\) is small enough.
Such a model is unfair to the elements of \(A\),
which can lead to legal and ethical issues if \(A\) is a marginalized group.
Similarly, a model that allows for the reconstruction of \(\xi\),
thereby betraying the privacy of the individuals in the training set
is also considered bad, even if it has excellent performance.
Alternatively, the data could be of low quality
due to its source rather than its content.
For example, a dataset that can include deliberately wrong information.
Accounting for these considerations, we amend \Cref{princple:performance_is_dq}
to a more general form.
\begin{principle}\label{princple:mq_is_dq}
	A good dataset \(\xi\) can be defined as one that (when used for \gls{abb:erm})
	produces a good model, which in turn can be defined
	as a model \(\hat{\theta}_{\xi}\) that satisfies certain ethical and social criteria,
	in addition to having low true risk \(\risk\left( \hat{\theta}_{\xi} \right)\).
\end{principle}
An extended version of the above taxonomy to include these extrinsic considerations
in compliance with \Cref{princple:mq_is_dq} is shown in \Cref{fig:dq:dims:taxonomy}.
Each dimension of this taxonomy will be discussed in detail in  \Crefrange{sec:independence}{sec:trust}.


\begin{figure}[hbt]
	\begin{center}
		% \resizebox{\linewidth}{!}{\subsection{Data quality dimensions}%
\label{sub:data_quality_dimensions}

It follows from \Cref{princple:performance_is_dq} that
the \emph{best} dataset \(\xi\) produces the \emph{best}
(or Bayes-optimal) model, i.e., \(\hat{\theta}_{\xi} = \theta^{\ast}\).
More generally, a \emph{good} dataset is one that produces a \emph{good} model,
i.e., \(\hat{\theta}_{\xi} \approx \theta^{\ast}\).
Early works on the theory of \gls{abb:ml}~\cite{%
	vapnikPrinciplesRiskMinimization1991,%
	neweyChapter36Large1994,%
	vapnikNatureStatisticalLearning2010%
}
have provided theoretical guarantees for this to be the case
under certain fairly general conditions
on the hypothesis class \(\mathcal{H}\) and the parameter space \(\Theta\).
More precisely, they have shown that, with high probability,
a large enough dataset \(\xi\), drawn \gls{abb:iid}
from the true distribution \(\Pr_{(X, Y)}\),
will produce a good model (in the sense defined above).

This already gives us three sufficient conditions for a dataset to be good,
namely that it is:
\begin{enumerate*}[label=(\roman*)]
	\item large enough,
	\item \gls{abb:iid}, and
	\item drawn from the true distribution.
\end{enumerate*}
Size is usually not seen as a data quality issue,
so we will not consider it further.
The other two conditions, however, are of great interest to us,
and can serve as the main axes of our definition of data quality.
Based on the previous analysis,
we can tentatively divide \gls{abb:dq} concerns into the following categories:
\begin{enumerate}
	\item \emph{Data dependence}.
	      Which occurs when the elements of \(\xi\) are not independent.
	\item \emph{Distribution mismatch}.
	      Which occurs when the elements of \(\xi\)
	      do not follow the true distribution \(\Pr_{(X, Y)}\).
	      The joint distribution \(\Pr_{(X, Y)}\)
	      is characterized by the marginal distribution \(\Pr_{X}\)
	      and the conditional distribution \(\Pr_{Y \mid X}\),
	      which allows us to distinguish between two types of distribution mismatch:
	      \begin{enumerate}
		      \item \emph{Attribute skew}.
		            Which happens when the sampling process
		            is biased towards certain regions of the input space \(\mathcal{X}\).
		      \item \emph{Label noise}.
		            Which happens when the labeling process is error-prone.
	      \end{enumerate}
\end{enumerate}
It is worth noting that the above scenarios are not mutually exclusive.
Distribution mismatch can happen due to a combination
of attribute skew and label noise,
and data dependence is likely to cause attribute skew.

An easy objection one could raise against the above categorization,
and indeed against \Cref{princple:performance_is_dq} itself,
is its reliance on the premise that model quality
can be reduced to performance (i.e., true risk).
In doing so, it fails to completely capture
the context-dependent nature of \gls{abb:dq}.
As a result, all the above considerations
are more or less \emph{intrinsic} to the data.

To justify the inclusion of \emph{extrinsic} considerations,
we argue that model performance is not sufficient to capture model quality.
A model can be considered bad while having excellent performance,
if it fails to satisfy certain ethical or social criteria.
For example, a model that has near-optimal true risk
can have arbitrarily high risk on a certain subset \(A \subset \mathcal{X}\),
if \(\Pr{\left[X \in A \right]}\) is small enough.
Such a model is unfair to the elements of \(A\),
which can lead to legal and ethical issues if \(A\) is a marginalized group.
Similarly, a model that allows for the reconstruction of \(\xi\),
thereby betraying the privacy of the individuals in the training set
is also considered bad, even if it has excellent performance.
Alternatively, the data could be of low quality
due to its source rather than its content.
For example, a dataset that can include deliberately wrong information.
Accounting for these considerations, we amend \Cref{princple:performance_is_dq}
to a more general form.
\begin{principle}\label{princple:mq_is_dq}
	A good dataset \(\xi\) can be defined as one that (when used for \gls{abb:erm})
	produces a good model, which in turn can be defined
	as a model \(\hat{\theta}_{\xi}\) that satisfies certain ethical and social criteria,
	in addition to having low true risk \(\risk\left( \hat{\theta}_{\xi} \right)\).
\end{principle}
An extended version of the above taxonomy to include these extrinsic considerations
in compliance with \Cref{princple:mq_is_dq} is shown in \Cref{fig:dq:dims:taxonomy}.
Each dimension of this taxonomy will be discussed in detail in  \Crefrange{sec:independence}{sec:trust}.


\begin{figure}[hbt]
	\begin{center}
		% \resizebox{\linewidth}{!}{\subsection{Data quality dimensions}%
\label{sub:data_quality_dimensions}

It follows from \Cref{princple:performance_is_dq} that
the \emph{best} dataset \(\xi\) produces the \emph{best}
(or Bayes-optimal) model, i.e., \(\hat{\theta}_{\xi} = \theta^{\ast}\).
More generally, a \emph{good} dataset is one that produces a \emph{good} model,
i.e., \(\hat{\theta}_{\xi} \approx \theta^{\ast}\).
Early works on the theory of \gls{abb:ml}~\cite{%
	vapnikPrinciplesRiskMinimization1991,%
	neweyChapter36Large1994,%
	vapnikNatureStatisticalLearning2010%
}
have provided theoretical guarantees for this to be the case
under certain fairly general conditions
on the hypothesis class \(\mathcal{H}\) and the parameter space \(\Theta\).
More precisely, they have shown that, with high probability,
a large enough dataset \(\xi\), drawn \gls{abb:iid}
from the true distribution \(\Pr_{(X, Y)}\),
will produce a good model (in the sense defined above).

This already gives us three sufficient conditions for a dataset to be good,
namely that it is:
\begin{enumerate*}[label=(\roman*)]
	\item large enough,
	\item \gls{abb:iid}, and
	\item drawn from the true distribution.
\end{enumerate*}
Size is usually not seen as a data quality issue,
so we will not consider it further.
The other two conditions, however, are of great interest to us,
and can serve as the main axes of our definition of data quality.
Based on the previous analysis,
we can tentatively divide \gls{abb:dq} concerns into the following categories:
\begin{enumerate}
	\item \emph{Data dependence}.
	      Which occurs when the elements of \(\xi\) are not independent.
	\item \emph{Distribution mismatch}.
	      Which occurs when the elements of \(\xi\)
	      do not follow the true distribution \(\Pr_{(X, Y)}\).
	      The joint distribution \(\Pr_{(X, Y)}\)
	      is characterized by the marginal distribution \(\Pr_{X}\)
	      and the conditional distribution \(\Pr_{Y \mid X}\),
	      which allows us to distinguish between two types of distribution mismatch:
	      \begin{enumerate}
		      \item \emph{Attribute skew}.
		            Which happens when the sampling process
		            is biased towards certain regions of the input space \(\mathcal{X}\).
		      \item \emph{Label noise}.
		            Which happens when the labeling process is error-prone.
	      \end{enumerate}
\end{enumerate}
It is worth noting that the above scenarios are not mutually exclusive.
Distribution mismatch can happen due to a combination
of attribute skew and label noise,
and data dependence is likely to cause attribute skew.

An easy objection one could raise against the above categorization,
and indeed against \Cref{princple:performance_is_dq} itself,
is its reliance on the premise that model quality
can be reduced to performance (i.e., true risk).
In doing so, it fails to completely capture
the context-dependent nature of \gls{abb:dq}.
As a result, all the above considerations
are more or less \emph{intrinsic} to the data.

To justify the inclusion of \emph{extrinsic} considerations,
we argue that model performance is not sufficient to capture model quality.
A model can be considered bad while having excellent performance,
if it fails to satisfy certain ethical or social criteria.
For example, a model that has near-optimal true risk
can have arbitrarily high risk on a certain subset \(A \subset \mathcal{X}\),
if \(\Pr{\left[X \in A \right]}\) is small enough.
Such a model is unfair to the elements of \(A\),
which can lead to legal and ethical issues if \(A\) is a marginalized group.
Similarly, a model that allows for the reconstruction of \(\xi\),
thereby betraying the privacy of the individuals in the training set
is also considered bad, even if it has excellent performance.
Alternatively, the data could be of low quality
due to its source rather than its content.
For example, a dataset that can include deliberately wrong information.
Accounting for these considerations, we amend \Cref{princple:performance_is_dq}
to a more general form.
\begin{principle}\label{princple:mq_is_dq}
	A good dataset \(\xi\) can be defined as one that (when used for \gls{abb:erm})
	produces a good model, which in turn can be defined
	as a model \(\hat{\theta}_{\xi}\) that satisfies certain ethical and social criteria,
	in addition to having low true risk \(\risk\left( \hat{\theta}_{\xi} \right)\).
\end{principle}
An extended version of the above taxonomy to include these extrinsic considerations
in compliance with \Cref{princple:mq_is_dq} is shown in \Cref{fig:dq:dims:taxonomy}.
Each dimension of this taxonomy will be discussed in detail in  \Crefrange{sec:independence}{sec:trust}.


\begin{figure}[hbt]
	\begin{center}
		% \resizebox{\linewidth}{!}{\input{assets/figures/tikz/dimensions}}
	\end{center}
	\caption{A taxonomy of data quality dimensions}
	\label{fig:dq:dims:taxonomy}
\end{figure}

}
	\end{center}
	\caption{A taxonomy of data quality dimensions}
	\label{fig:dq:dims:taxonomy}
\end{figure}

}
	\end{center}
	\caption{A taxonomy of data quality dimensions}
	\label{fig:dq:dims:taxonomy}
\end{figure}

}
	\end{center}
	\caption{A taxonomy of data quality dimensions}
	\label{fig:dq:dims:taxonomy}
\end{figure}

