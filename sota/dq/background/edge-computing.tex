\subsection{Edge computing}\label{sub:edge_computing}

The edge of the network (or simply the edge) can be defined as the interface
between the outer layers of the network,
represented by user devices commonly referred to as \emph{end devices}
(e.g., sensors, smartphones, and laptops),
and its \emph{core} of the network (often represented by a cloud server)%
~\cite{singhEdgeAISurvey2023}.
This definition can be vague, as the internet does not come with a clear predefined hierarchy.
As a result, the terminology used to describe the edge is often ambiguous,
particularly when it comes to the distinction
between \emph{end devices} and \emph{edge nodes}%
~\cite{wangEdgeAIConvergence2020}.
In this work, we adopt the terminology proposed by~\citet{wangEdgeAIConvergence2020}:

\begin{enumerate}
	\item \emph{End devices} or \emph{end nodes}, are the bottom layer of the network.
	      Any mobile device (e.g., \gls{abb:iot} device, smartphone, or \gls{abb:av})
	      with some (usually limited) computational power can be considered an end device.
	      Due to their limited computational power,
	      end devices are only capable of performing the most basic \gls{abb:ml} tasks,
	      such as inference.
	\item \emph{Edge nodes}, edge servers, or edge devices,
	      are the middle layer of 3-tiered end-edge-cloud architecture.
	      They are usually more powerful than end devices,
	      and are thus capable of performing more complex \gls{abb:ml} tasks,
	      such as training or orchestration.
\end{enumerate}

A more fine-grained hierarchy for the network graph is given in \Cref{fig:edge_architecture}.
Where the core is made up of the 2 innermost layers,
the edge is the \nth{3} layer, and end devices occupy the next layer.
\begin{figure}[hbt]
	\begin{center}
		% \includegraphics[width=.8\linewidth]{assets/figures/drawio/pdf/edge-architecture.pdf}
	\end{center}
	\caption{Network architecture for edge computing~\cite{zhouEdgeIntelligencePaving2019}.}%
	\label{fig:edge_architecture}
\end{figure}

The difficulty of defining and agreeing on a clear hierarchy for the edge
is partially behind the emergence of multiple paradigms for \gls{abb:ec}.
These include \emph{cloudlet computing} and \gls{abb:mec},
which rely on a 3-tiered architecture,
and \gls{abb:fc},
which organizes the network into an end-to-cloud \emph{continuum} rather than a hierarchy%
~\cite{singhEdgeAISurvey2023}.
