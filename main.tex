\documentclass[12pt]{report}

\usepackage[svgnames,dvipsnames]{xcolor}
\usepackage[T1]{fontenc}
\usepackage{lmodern} % remove some warnings
\usepackage{graphicx}
\usepackage{amsmath,amsfonts,amssymb,amsthm}
\usepackage{pifont} % for checkmark and crossmark
\usepackage{mathtools,thmtools}
% \usepackage[font=small]{subfig}
\usepackage{subcaption}
\usepackage{wrapfig}
\usepackage[normalem]{ulem}
\usepackage{ifthen}
\usepackage{booktabs}
\usepackage{capt-of}
\usepackage{footnote}
\usepackage{caption}
\usepackage{upgreek}
\usepackage{gensymb}
\usepackage{float}
\usepackage{geometry}
\usepackage[inline]{enumitem}
\usepackage{import}
\usepackage{stmaryrd}
\usepackage[super]{nth}
\usepackage{bbm}
\usepackage{textcomp}% to get rid of warnings about \perthousand and \micro
\usepackage{adjustbox}
\usepackage{stackengine}% for \stackanchor
\usepackage{complexity}
\usepackage{tabularx}% for table linebreaks
\usepackage[toc,page]{appendix}
\usepackage{pdfpages}


\subimport{preamble}{algorithm}
\subimport{preamble}{abbreviations}
\subimport{preamble}{drawing}
\subimport{preamble}{notation}
\subimport{preamble}{theorems}
\subimport{preamble}{hyperref}
% \subimport{preamble}{code}

\geometry{a4paper, left=22mm,  top=22mm, bottom=22mm, right=22mm}
\hyphenpenalty=10000


%% The following content must be adapted for the final version


\title{Investigating the impact of data quality on the geometry and dynamics of machine learning}
\author{Mohammed D. Belgoumri}

\graphicspath{{assets/figures/}}
\begin{document}
\begin{titlepage}
	\addcontentsline{toc}{chapter}{Title Page}
	\thispagestyle{empty}
	\includegraphics[width=0.25\textwidth]{images/logo.jpeg}
	\begin{center}
		\vspace*{4cm}
		{\LARGE Data quality through the lens of loss landscape analysis}
		\vspace{3cm}
		\begin{large}


			{\bf Submitted as Research Report in SIT723/SIT724}
			\vspace{1cm}

			{\bf SUBMISSION DATE} \\
			T2-2024

			\vspace{3cm}
			\textbf{Mohammed D. Belgoumri}\\
			STUDENT ID 1234567 \\
			COURSE - Doctor of Philosophy (S700)
			\vfill

			{\bf
				\normalsize Supervised by: Dr. Mohamed R. Bouadjenek, Dr. Sunil Aryal, Dr. Hakim Hacid
			}\\
		\end{large}
	\end{center}
\end{titlepage}

\begin{abstract}
	\addcontentsline{toc}{chapter}{Abstract}
	\thispagestyle{plain}
	\pagenumbering{roman}
	The quality of a \glsxtrfull{abb:ml} model can be measured on three (non-orthogonal) axes:
	\begin{enumerate*}[label=(\arabic*)]
		\item approximation (or expressivity),
		\item generalization, and
		\item optimization.
	\end{enumerate*}
	Of these three, training data exerts significant influence on two,
	namely, generalization and optimization,
	justifying the centrality of \glsxtrfull{abb:dq} as a concern in \glsxtrshort{abb:ml}.
	While the interplay between data quality and model performance
	is both intuitive and well-documented as a phenomenon, the
	mechanisms that underlie it are much less well-understood.
	The above observation is even more pressing in distributed settings,
	such as \glsxtrfull{abb:fl} on the edge,
	both because \glsxtrshort{abb:dq} is harder to assess in such settings,
	and because distributed training complicates the interaction between data and model.
	In light of this situation, we undertake the task of characterizing the impact of data quality
	on the learning dynamics of \glsxtrfull{abb:dl},
	in the general context and in the specific context of \glsxtrshort{abb:ml} on the edge.
	We attempt to do so through the lens of loss landscape analysis,
	a universally applicable framework for studying \glsxtrfull{abb:nn} phenomena.
	Our objective is first and foremost, to provide an improved understanding
	of the ways in which data quality influences the learning process, and by extension,
	the mechanisms by which existing \glsxtrshort{abb:dq}-robsustness techniques work.
	Our study is therefore exploratory in nature, and more descriptive than prescriptive.
	That being said, we do find multiple opportunities for the development of new techniques
	for improving, rather than merely understanding \glsxtrshort{abb:dq}-robustness.
	%
	We start by reviewing the literature on \glsxtrfull{abb:dq} in edge \glsxtrshort{abb:ml},
	compiling our findings in a survey that is currently under review
	for publication in the \glsxtrfull{abb:vldb} journal (see \Cref{app:survey}).
	Next, we take a closer look at the literature on loss landscape analysis,
	with particular focus on data-centric approaches,
	with the aim of identifying gaps in our current understanding.
	We then formulate a list of research questions and objectives and describe our methodology.
	Finally, we showcase the current state of our advancement,
	providing an account of the results we have obtained so far,
	a discussion of the challenges we are facing,
	a preview of our next steps,
	and a summary of our failed attempts.
\end{abstract}
\setcounter{page}{2}
\tableofcontents
\newpage
\listoffigures
\listoftables

\setlength{\parskip}{1em}
\renewcommand{\baselinestretch}{2.0}

\newpage
\pagenumbering{arabic}
\setcounter{page}{1}

\chapter*{Introduction}
\import{sota}{sota}

% \chapter{Motivation}
% \section{ML, FL, and data quality}
% \section{Loss landscape analysis}
% \chapter{State of the art}
% \chapter{Contribution}
% \section{Research questions}
% \section{Methodology}
% \section{Results and discussion}
% \section{Failed attempts}
\chapter*{Conclusion}
\bibliography{references}
% \begin{appendices}
% 	\chapter{Data quality for edge ML survey}\label{app:survey}
% 	\includepdf[pages=-21]{appendices/survey.pdf}
% \end{appendices}
\end{document}